\documentclass[11pt]{beamer}
\usetheme{Warsaw}
\usepackage[utf8]{inputenc}
\usepackage[german]{babel}
\usepackage[T1]{fontenc}
\usepackage{amsmath}
\usepackage{amsfonts}
\usepackage{amssymb}
\usepackage{tikz}
\usetikzlibrary{shapes.multipart}
\usetikzlibrary{shapes,arrows}
\usepackage{hyperref}

\usepackage{soul}
\author{Michael F. Schönitzer}
\title{Overlayfs}

%\setbeamercovered{transparent} 
%\setbeamertemplate{navigation symbols}{} 
%\logo{} 
\institute{OpenSourceTreffen München} 
%\date{} 
%\subject{} 

\newcommand{\pfeil}{$\qquad\rightarrow\;$}
\newcommand{\pfeilk}{$\rightarrow\;$}


\begin{document}

\begin{frame}
\titlepage
\end{frame}

\section{Union Mounts}
\begin{frame}{Union Mounts / Overlaying Filesystems}

\begin{figure}[tbp]
\center
\begin{tikzpicture}[
  single/.style={draw, dashed, text centered, align=center, anchor=text, text centered, rounded corners, inner ysep=1ex, inner xsep=1ex},
  double/.style={draw, solid, draw, fill=blue!20, anchor=text, text width=7em, text centered, rounded corners, inner ysep=2em, rectangle split, rectangle split parts=2},
  triple/.style={draw, anchor=text, rectangle split,rectangle split parts=3}
  ]
  \node[single] {
  merged/\\
  
  \tikz{\node[double] {
     upper/
    \nodepart{second}
      lower/
  };}
  };
\end{tikzpicture}
\end{figure}
\end{frame}

\begin{frame}{Union / Overlaying Filesystems}
\begin{itemize}
\item 1993: Inheriting File System (IFS)
\item 2003: unionfs
\item 2006: aufs
\item 2012: overlayfs
\item 2014: overlayfs im Kernel \textbf{3.18}
\end{itemize}

\end{frame}

\section{Overlayfs}
\begin{frame}{Overlayfs: Einschränkungen}
\begin{itemize}
\item lowerdir
  \begin{itemize}
  \only<1>{\item Beliebiges Dateisystem (incl. overlayfs)}
  \only<2->{\item\st{Beliebiges Dateisystem (incl. overlayfs)}}
  \item<2-> Nicht Caseinsensitive, automount, revalidate \\
    \pfeil nicht fat32, NFS, ISO 9660/Joliet…
  \end{itemize}
\pause\pause

\item upperdir
  \begin{itemize}
  \item Dateisystem mit bestimmten Features \\
   \pfeil nicht NFS, fat32, ntfs…
  \end{itemize}
\pause

\item workdir
  \begin{itemize}
  \item leer
  \item selbes Dateisystem wie upperdir
  \end{itemize}
\end{itemize}
  
\end{frame}


\begin{frame}{Overlayfs: Funktionsweise}
\begin{itemize}
 \item Datei/Ordner nur im unterem Verzeichnis \\ \pfeil zeige diese
 \item Datei/Ordner nur im oberen Verzeichnis \\ \pfeil  zeige diese
 \item Datei in beiden Verzeichnissen \\ \pfeil  zeige obere
 \item Ordner in beiden Verzeichnissen \\ \pfeil merge, Metadaten vom oberen
 \item Änderungen im Ziel \\ \pfeil werden im oberen durchgeführt
 \\ \pfeil Whiteouts \hfill\pfeilk Device-Datei mit Nummer 0/0 
 \\ \pfeil Opaque directories \hfill\pfeilk xattr trusted.overlay.opaque=y
\end{itemize}
\end{frame}

\begin{frame}
\begin{block}{Verwendung:}\footnotesize
mount -t overlay overlay \textbackslash \\ $\qquad\quad$
-olowerdir=/lower,upperdir=/upper,workdir=/work /merged
\end{block}
\vfill

\begin{block}{fstab:}\footnotesize
overlay /merged/ overlay lowerdir=/lower,upperdir=/upper,workdir=/work 0 0
\end{block}
\vfill

\begin{block}{Multiple lower layers (ab Kernel 4.0)}\footnotesize
mount -t overlay overlay -olowerdir=/lower1:/lower2:/lower3 /merged
\end{block}
\end{frame}

\section{Overlayroot}
\begin{frame}

\begin{block}{Kombination mit}
bind-mounts, chroot, cgroups, unshare (namespaces)
\end{block}
\pause

\begin{figure}[tbp]
\center
\tikzstyle{line} = [draw, -latex']
\begin{tikzpicture}[node distance = 5cm, auto, align=center, text centered, rounded corners,  
  souround/.style={draw, dashed, anchor=text, inner ysep=1ex, inner xsep=1ex},
  single/.style={draw, solid, anchor=text, inner ysep=1ex, inner xsep=1ex, node distance = 1cm},
  single2/.style={draw, densely dotted, anchor=text, inner ysep=1ex, inner xsep=1ex, node distance = 1cm},
  double/.style={draw, solid, fill=blue!20, anchor=text, text width=7em, inner ysep=2em, rectangle split, rectangle split parts=2}  ]
  \node (foo) {
  
  \tikz{
  \node[single] (container)  {
  containerfile
  };
  }
\\ \\ \\};
  
  \node[souround, right of=foo] {
  newroot/\\
  
  \tikz{\node[double] (upper) {
     upper/
    \nodepart{second}
      /
  };}
  
  
  \tikz{
  \node[single2] (proc)  {  proc  };
  \node[single2, below of=proc] (sys)  {  sys  };
  \node[single2, below of=sys] (dev)  {  dev  };
  \node[single2, below of=dev] (run)  {  run  };
  
  }

  };
  
  \path [line] (1.4,0.75) -- +(1,0);

\end{tikzpicture}
\end{figure}
\pause


\begin{block}{overlayroot.sh}
\url{https://github.com/Nudin/overlayfsfun}
\end{block}

\end{frame}

\begin{frame}
\center{\Huge
Happy Hacking}

\vfill
{\scriptsize\begin{block}{Hinweis:}
Ich such grade einen Job oder Praktikum im Linux-Umfeld. Hinweise gerne gesehen.
\end{block}
\vfill
\includegraphics[scale=0.4]{cc} \\
Alle Inhalte dieses Vortrags sind unter \href{https://creativecommons.org/licenses/by-sa/4.0/deed.de}{CC-BY-SA} 4.0.
}
\end{frame}



\end{document}